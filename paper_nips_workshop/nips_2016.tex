\documentclass{article}

% if you need to pass options to natbib, use, e.g.:
% \PassOptionsToPackage{numbers, compress}{natbib}
% before loading nips_2016
%
% to avoid loading the natbib package, add option nonatbib:
% \usepackage[nonatbib]{nips_2016}

%\usepackage{nips_2016}

% to compile a camera-ready version, add the [final] option, e.g.:
\usepackage[final]{nips_2016}

\usepackage[utf8]{inputenc} % allow utf-8 input
\usepackage[T1]{fontenc}    % use 8-bit T1 fonts
\usepackage{hyperref}       % hyperlinks
\usepackage{url}            % simple URL typesetting
\usepackage{graphicx} % more modern
\usepackage{booktabs}       % professional-quality tables
\usepackage{amsfonts}       % blackboard math symbols
\usepackage{nicefrac}       % compact symbols for 1/2, etc.
\usepackage{microtype}      % microtypography
\usepackage{floatrow}
\usepackage{amsmath}
\usepackage{amssymb}

\newcommand{\ica}{\hspace{0.15cm}}
\renewcommand{\arraystretch}{1.00}
\newfloatcommand{capbtabbox}{table}[][\FBwidth]

\title{GANS for Sequences of Discrete Elements \\with the Gumbel-softmax Distribution}

% The \author macro works with any number of authors. There are two
% commands used to separate the names and addresses of multiple
% authors: \And and \AND.
%
% Using \And between authors leaves it to LaTeX to determine where to
% break the lines. Using \AND forces a line break at that point. So,
% if LaTeX puts 3 of 4 authors names on the first line, and the last
% on the second line, try using \AND instead of \And before the third
% author name.

\author{
    Matt Kunser\\
    Alan Turing Institute
    \And
    Jos\'e Miguel Hern\'andez-Lobato\\
    University of Cambridge
  %% examples of more authors
  %% \And
  %% Coauthor \\
  %% Affiliation \\
  %% Address \\
  %% \texttt{email} \\
  %% \AND
  %% Coauthor \\
  %% Affiliation \\
  %% Address \\
  %% \texttt{email} \\
  %% \And
  %% Coauthor \\
  %% Affiliation \\
  %% Address \\
  %% \texttt{email} \\
  %% \And
  %% Coauthor \\
  %% Affiliation \\
  %% Address \\
  %% \texttt{email} \\
}

\begin{document}
% \nipsfinalcopy is no longer used

\maketitle

\begin{abstract}
Generative Adversarial Networks (GAN) have liminations when the goal is to
generate sequences of discrete elements. The reason for this is that
samples from a distribution on discrete objects such as the multinomial are
not differentiable with respect to the distribution parameters. This problem
can be avoided by using the Gumbel-softmax distribution, which is a continuous
approximation to a multinomial distribution parameterized in terms of the
softmax function. In this work, we evaluate the performance of GANs based on
recurrent neural networks with Gumbel-softmax output distributions in the task
of generating sequences of discrete elements.
\end{abstract}

\section{Introduction}

Generative adversarial networks (GANs) are methods for generating synthetic
data with similar statistical properties as the real one
\cite{goodfellow2014generative}. In the GAN methodology a discriminative neural network D is
trained to distinguish whether a given data instance is synthetic or real,
while a generative network G is jointly trained to confuse D by generating high
quality data. This approach has been very sucessful in computer vision tasks for
generating samples of natural images
\cite{denton2015deep,dosovitskiy2016generating,radford2016}.

GANs work by propagating gradients back from the discriminator D through the
generated samples to the generator G. This is perfectly feasible when the
generated data is continuous such as in the examples with images mentioned
above. However, a lot of data exists in the form of squences of discrete items.
For example, text sentences \cite{Bowman2016}, molecules encoded in the SMILE language \cite{gomez2016automatic}, etc. In these
cases, the discrete data is not differentiable and the backpropagated gradients
are always zero. 

Discrete data, encoded using a one-hot representation, can be sampled from a
multinomial distribution with probabilities given by the output of a softmax
function. The resulting sampling process is not differentiable.  However, we can obtain
a differentiable approximation by sampling from the Gumbel-softmax distribution
\cite{jang2016categorical}. This distribution has been previously used to train
variatoinal autoencoders with discrete latent variables \cite{jang2016categorical}. Here, we propose to
use it to train GANs on sequences of discrete tokens and we evaluate its
performance in this setting.



\section{Gumbel-softmax distribution}

The Gumbel softmax distribution 

The softmax function can be used to parameterized a multi-nomial distribution
on a one-hot-encoding $d$-dimensional vectors $\mathbf{y}$ in terms of a
continuous $d$-dimensional vector $\mathbf{x}$ as follows:
\begin{align}
p(y_i=1) = \frac{\exp(\mathbf{x}_i)}{\sum_{j=1}^K\exp(\mathbf{x}_j)}\,.
\end{align}
You can show that this generative process for $\mathbf{y}$ is the same as
\begin{align}
\mathbf{y} = \text{one\_hot}(\underset{i}{\arg\max} (x_i + z_i))\,,\label{eq:1}
\end{align}
where the $z_i$ are independent and follow a Gumbel distribution with zero
location and unit scale.

The sample generated in (\ref{eq:1}) has gradient zero with respect to
$\mathbf{x}$ because the $\text{one\_hot}(\underset{i}{\arg\max}(\cdot)$
operator is not differentiable.

We propose to approximate this operator with a differentiable function based on the soft-max function.
In particular, we approximate $\mathbf{y}$ with 
\begin{align}
\mathbf{y} = \text{softmax}(1 / \tau (\mathbf{x} + \mathbf{z})))\,,\label{eq:2}
\end{align}






\section{A recurrent neural network for discrete sequences}

%!TEX root=nips_2016.tex
\section{Experiments}
We now show the power of our adversarial modeling framework for generating discrete sequences. To illustrate this we consider modeling the context-free grammar introduced in Section 3. We generate $5000$ samples with a maximum length of $12$ characters from the context-free grammar (CFG) for our training set. We pad all sequences with less than $12$ characters with spaces.

\subsection*{Optimization details}
We train both the discriminator and generator using ADAM \cite{kingma2014adam} with a fixed learning rate of $0.001$ and a mini-batch size of $m\!=\!200$. Inspired by the work of \cite{sonderby2016amortised} who use input noise to stabilize GAN training, for every input $\x$ we form a vector $\mathbf{h}$ such that its softmax (instead of being one-hot) places a probability of approximately $0.9$ on the correct character and a probability of $(1-0.9)/(d-1)$ on the remaining $d\!-\!1$ characters. We then apply the Gumbel-softmax trick to generate a vector $\mathbf{y}$ as in equation~(\ref{sec:gumbel:eq:2}). We use this vector instead of $\x$ throughout training. We train the generator and discriminator for $20,000$ mini-batch iterations. During the training we linearly anneal the temperature of the Gumbel-softmax distribution, from $\tau\!=\!5$ (i.e., a very flat distribution) to $\tau\!=\!1$ (a more peaked distribution) for iterations $1$ to $10,000$ and then kept at $\tau\!=\!1$ until training ends. 


\begin{figure*}[t!]
\begin{center}
\centerline{\includegraphics[width=\textwidth]{gan_losses.pdf}}
\vspace{-2ex}
\caption{The generative and discriminative losses throughout training. Ideally the loss of the discriminator should increase while the generator should decrease as the generator becomes better at mimicking the real data. \textbf{(a)} The default network with Gumbel-softmax temperature annealing. \textbf{(b)} The same setting as (a) but increasing the size of the generated samples to $1,000$. \textbf{(c)} Only varying the input vector temperature. \textbf{(d)} Only introducing random noise into the hidden state and not the cell state.}
%\vspace{-5ex}
\label{figure.losses}
\end{center}
\end{figure*}

\subsection*{Learning a CFG}
Figure~\ref{figure.losses} \textbf{(a)} shows the generator and discriminator losses throughout training for this setting. We experimented with increasing the size of the generated samples to $1,000$, as this has been reported to improve GAN modeling \cite{huszar2015not}, shown in Figure~\ref{figure.losses} \textbf{(b)}. We also experimented with just varying the temperature for the input vectors $\mathbf{y}$ and fixing the generator temperature to $\tau\!=\!1$ (in Figure~\ref{figure.losses} \textbf{(c)}).  Finally, we also tried just introducing random noise into the hidden state and allowing the network to learn an initial cell state $C_0$ (Figure~\ref{figure.losses} \textbf{(d)}).

\begin{figure*}[t!]
\begin{center}
\centerline{\includegraphics[width=\textwidth]{generated.pdf}}
\vspace{-2ex}
\caption{The generated text for MLE and GAN models. The plots \textbf{(a)}-\textbf{(d)} correspond to the models of Figure~\ref{figure.losses}.}
%\vspace{-5ex}
\label{figure.generate}
\end{center}
\end{figure*}

Figure~\ref{figure.generate} shows the text generated by MLE and GAN models. Each row is a sample from either model, each consisting of $12$ characters (we have included the blank space character as some training inputs are padded with spaces if less than $12$ characters). While the MLE LSTM is not strictly a generative model in the sense of drawing a discrete sequence from a distribution, we include it for reference. We can see that our GAN models are learning to generate alternating sequences of $x$'s, similar to the MLE result. Specifically, the 4th, 10th, and 17th rows of plot \textbf{(a)}, show samples that are very close to the training data, and many such examples exist for the remaining plots as well. 

We believe that these results, as a proof of concept, show strong promise for training GANs to generate discrete sequence data. Further, we believe that incorporating recent advances in GANs such as training GANs using variational divergence minimization \cite{nowozin2016f} or via density ratio estimation \cite{uehara2016generative} could yield further improvements. We aim to experiment with these in future work.


\bibliography{references}
\bibliographystyle{plain}

\end{document}
