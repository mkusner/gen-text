\section{Introduction}

Generative adversarial networks (GANs) are methods for generating synthetic
data with similar statistical properties as the real one
\cite{goodfellow2014generative}. In a GAN, a discriminative neural network D is
trained to distinguish whether a given data instance is synthetic or real, while a
generative network G is jointly trained to confuse D by generating high quality
data. This approach has been very sucessful in computer vision tasks of
generating samples of natural images \cite{denton2015deep,dosovitskiy2016generating,radford2016}.

They


one is an important problem in unsupervised learning

shown to  generative models
in different domains.

are 

A task specific loss may not be directly
available

